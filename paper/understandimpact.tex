\section{Evaluating the Impact}

\newcommand{\jt}{Joda-Time\xspace}

\newcommand{\jfreecharttests}{2234\xspace}%change the total num
\newcommand{\jodatimetests}{3875\xspace}
\newcommand{\xmlsecuritytests}{108\xspace}
\newcommand{\crystaltests}{75\xspace}
\newcommand{\synoptictests}{118\xspace}
\newcommand{\totaltests}{4176\xspace}

\newcommand{\jfreechartautotests}{2946\xspace}
\newcommand{\jodatimeautotests}{2639\xspace}
\newcommand{\xmlsecurityautotests}{665\xspace}
\newcommand{\crystalautotests}{3198\xspace}
\newcommand{\synopticautotests}{2467\xspace}
\newcommand{\totalautotests}{8969\xspace}

\label{sec:impact}

This section describes our empirical evaluation of
the impact of test dependence on three types of
downstream testing techniques.
We first briefly present the necessary background of
each downstream testing technique (Section~\ref{sec:evalbackground}), then
present the evaluation methodology (Section~\ref{sec:evalmethod}), and
show the results (Section~\ref{sec:evalresults}).


\subsection{Downstream Testing Techniques}
\label{sec:evalbackground}
Test prioritization, selection, and parallelization are three
important downstream testing techniques. These three techniques may
change the execution order of an existing test suite in
certain ways to make it run
faster, achieve coverage faster, detect faults earlier,
or fulfill other testing goals.

\subsubsection{Test prioritization.}

Test prioritization schedules test cases
for execution in an order that attempts to
increase their effectiveness.
One application of prioritization techniques involves
regression testing --- the retesting of software following modifications;
in this context, prioritization techniques can take advantage of information gathered about the previous
execution of test cases to obtain test case orderings.

Test prioritization techniques developed in the literature
fall into three major categories: (1) techniques
that order test cases based on their total coverage of
code components; (2) techniques that order test
cases based on their coverage of code components
not previously covered; (3) techniques that order test
cases based on their estimated ability to reveal faults
in the code components that they cover. In addition to the
test execution information, the third
category requires a comprehensive history of known
faults, which are often absent in practice.
% and
%approximated by seeded faults~\cite{}.


\begin{figure}
\centering
\setlength{\tabcolsep}{0.25\tabcolsep}
\begin{tabular}{|l|l|}
%\toprule
\hline
\textbf{Label} & \textbf{Technique Description} \\
\hline
T1 & Randomized ordering \\
T3 & Prioritize on coverage of statements \\
T4 & Prioritize on coverage of statements not yet covered\\
T5 & Prioritize on coverage of methods\\
T7 & Prioritize on coverage of functions not yet covered \\
%\bottomrule
\hline
%\textbf{Total}& &  & &  \\ 
%\hline
\end{tabular}
\caption{Five test prioritization techniques used
to assess the impact of dependent tests. These five
techniques are introduced in Table 1
of~\cite{Elbaum:2000:PTC:347324.348910}. (We use
the same labels as in~\cite{Elbaum:2000:PTC:347324.348910}. We did not
implement the other 9 test prioritization techniques
introduced in~\cite{Elbaum:2000:PTC:347324.348910}, since
they require a fault history that is not
available for our subject programs.)
}
\label{tab:testprio}
\end{figure}




In this paper, we focus on evaluating 5 well-known test prioritization
techniques from the first and second categories (summarized
in Figure~\ref{tab:testprio}).
\todo{evalaute combination of test prioritization and selection}

%We wish to empirically investigate the validity of this unverified
%conventional wisdom, in order to understand whether
%test dependence can affect the results of test prioritization
%techniques.


\subsubsection{Test selection.}

Test selection techniques are often designed for improving
regression testing to detect whether new defects
have been introduced into previously tested code and
to provide confidence that modifications
are correct. 
%Since regression testing is an expensive process,
%researchers have proposed regression test selection (for short,
%test selection) techniques as a way to reduce some of this expense.
In general, test selection techniques attempt to select and execute
only a subset of the original test cases in a program to
reduce the testing cost.


%Since test selection techniques execute only a subset of the original
%test suite, they 
Test selection techniques may change the execution
environment (i.e., tests running before the
selected test) of each selected test. 
Test selection can also be used together with
test prioritization by prioritizing the selected
subset of tests~\cite{}. Thus, the test independence
assumption becomes particularly important to keep the
selected tests behaving the same as in the original test suite.

In this paper, we focus on evaluating 2 well-known test
selection techniques (listed
in Figure~\ref{tab:testsel}). We conduct empirical evaluations
to investigate whether test dependence will affect their results.
For each test selection technique,
we also evalaute its effectiveness when combined with each test
prioritization technique listed in Figure~\ref{tab:testprio}.

\begin{figure}
\centering
\setlength{\tabcolsep}{0.25\tabcolsep}
\begin{tabular}{|l|l|}
%\toprule
\hline
\textbf{Label} & \textbf{Technique Description} \\
\hline
S1 & Select tests covering new/deleted/modified statements\\
S2 & Select tests covering new/deleted/modified methods\\
%\bottomrule
\hline
%\textbf{Total}& &  & &  \\ 
%\hline
\end{tabular}
\caption{Two test selection techniques used
to assess the impact of dependent tests. These
techniques are introduced in~\cite{}. These
two techniques form the basis of many other
popular test selection techniques~\cite{}.
}
\label{tab:testsel}
\end{figure}

\subsubsection{Test parallelization.}

Test execution parallelization (for short, test parallelization)
schedules the input tests for execution across
multiple CPUs to reduce the test time.
Such techniques are well adopted in
industry. For example, Visual Studio 2010 (and later)
supports a model of executing tests in parallel on a multi-CPU/core machine~\cite{}.
\todo{more evidence here}

Scheduling tests to multiple machines may change the execution
environment of each test and then affect the test result.
%Similarly, the test independence assumption is also critical to
%test execution parallelization, since  
%However, this critical
%assumption still remains unverified: to the best of our
%knowledge, all existing test parallelization techniques and
%tools~\cite{} implicitly assume that each test in a test
%suite is independent from one another.
This paper focuses on \parnum test parallelization techniques
(listed in Figure~\ref{tab:testpar}) and empirically investigates
the impacts of test dependence on them.

\begin{figure}
\centering
\setlength{\tabcolsep}{0.25\tabcolsep}
\begin{tabular}{|l|l|}
%\toprule
\hline
\textbf{Label} & \textbf{Technique Description} \\
\hline
P1 & Parallelize on test id\\
P2 & Random parallelization\\
P3 & Parallelize on test execution time\\
%\bottomrule
\hline
%\textbf{Total}& &  & &  \\ 
%\hline
\end{tabular}
\caption{Three test parallelization techniques used
to assess the impact of dependent tests. These
techniques are supported in industrial-strength
tools~\cite{}. \todo{explain each technique here.}
Each parallelization technique parameterized
by the number of available machines: $k$. We evaluate
each technique with $k$ = 2, 4, 8, and 16.
}
\label{tab:testpar}
\end{figure}



\subsection{Subject Programs}
\label{sec:subj}

Figure~\ref{tab:subjects} lists the programs and
tests used in our evaluation. We used these subject
programs because they have been developed for
a considerable amount of time (3--10 years) and each
of them includes a well-written unit test suite.

\jt~\cite{jodatime} is an open source
date and time library. It is a mature project that
has been under active development
for ten years. XML Security~\cite{xmlsecurity}
is a component library implementing XML signature and encryption
standards. XML Security is included in
the SIR repository~\cite{sir} and has been used widely
as a subject program in the software testing community.
Crystal~\cite{crystal} is a tool that
pro-actively examines developers' code and
identifies textual, compilation, and behavioral conflicts.
Synoptic~\cite{synoptic} is a tool to mine a finite state
machine model representation of a system from logs.
All of the subject programs' test suites are designed to be executed in
a single JVM, rather than requiring separate processes per test case~\cite{vmvm}.

Given the increasing importance of automated test generation
tools~\cite{PachecoLET2007, ZhangSBE2011, Csallner:2004, fraseretal:ISSTA:2011},
we also want to investigate dependent tests in automatically-generated
test suites. For each subject program, we use
Randoop~\cite{PachecoLET2007}, a state-of-the-art automated
test generation tool, to create a suite of 5,000 tests.
Randoop automatically drops textually-redundant tests 
and outputs a subset of the generated tests as
shown in Figure~\ref{tab:subjects}.

We discarded the automatically-generated test suite of
\jt, since many tests in it are non-deterministic ---
they depend on the current time.

\todo{explain two versions}
\todo{explain why these subject programs}

\subsection{Methodology}
\label{sec:evalmethod}

We implemented all testing techniques listed
in Figure~\ref{},~\ref{}, and~\ref{}, and evalaute
each technique on our subject programs in
Figure~\ref{}.

For each subject program, we first execute its
test suite in the \textit{default} order and record
the execution result of each test.
We adopt the results of the default
order of execution of a test suite as the expected results; these
are the results that a developer sees when running the suite
in the standard way. 

A test prioritization technique outputs a reordered test suite.
We execute the reordered test suite and count the number
of dependent tests that return different results
in the prioritized roder as they do when executed in the
unprioritized roder.

A test selection technique identifies a subset of the
input test suite to run during regression testing, but does not
change the relative order among the test suite. We run execute
the selected subset and count the number of dependent tests
that return different results as they do when executed
in the original test suite. In our experiment, for each
subject program in Figure~\ref{}, we use the next revision
as the ``new'' program version, and apply the test selection
technique to select a subset of the test suite that should
be re-run.

A test parallelization technique divides the input
test suite into multiple subsets, and schedules each
subset for execution on a different machine. Similar to
test selection, test parallelization techniques usually
do not change the relative order among the test suite.
We count the number of dependent tests that return different
results as they do when executed in the original test suite.
Our experiment focuses on the execution result of each test.
Due to resource limits, we run all subsets of the test suite
on a single machine one by one. Before running a subset,
we completely re-initialize the execution environment and launch
a new JVM.

\subsection{Results}
\label{sec:evalresults}

This section presents the evaluation results
for test prirotization, test selection, and
test parallelization techniques.

\subsubsection{Impact on Test Prioritization}

\begin{table}
\centering
\setlength{\tabcolsep}{1.25\tabcolsep}
\begin{tabular}{|l|l|l|l|l|l|}
%\toprule
\hline
\textbf{Subject Program} & T1 & T3 & T4 & T5 & T7 \\
\hline
\multicolumn{6}{|l|}{}  \\
\multicolumn{6}{|l|}{\textbf{Human-written Test Suites}}  \\
\hline
\jt& 0 & 0 & 1 & 0 & 0\\
XML Security& 0 & 0 & 0 & 0 & 0 \\
Crystal& 12 & 11 & 16 & 11 & 12 \\
Synoptic& 0 & 0 & 0 & 0 & 0 \\
JFreechart& 0 & 3 & 3 & 3 & 0 \\
%\bottomrule
\hline
\textbf{Total} & 12 & 14 & 20 & 14 & 12\\
\hline
\multicolumn{6}{|l|}{}  \\
\multicolumn{6}{|l|}{\textbf{Automatically-generated Test Suites}}  \\
\hline
\jt& 258 & 289 & 269 & 291 & 286\\
XML Security& 82 & 77 & 57 & 75 & 54 \\
Crystal& 79 & 74 & 86 & 89 & 90 \\
Synoptic& 0 & 2 & 3 & 3 & 3 \\
JFreechart& 3 & 5 & 5 & 5 & 5 \\
%\bottomrule
\hline
\textbf{Total} & 422 & 447 & 420 & 463 & 438\\
\hline
%\textbf{Total}& &  & &  \\ 
%\hline
\end{tabular}
\caption{Results of evaluating the \prionum test prioritization techniques
in Figure~\ref{tab:testprio} on four human-written unit test suites.
Each cell shows the number of dependent tests
that do not return the same results as they do when executed
in the default, unprioritized order. \todo{revise the number here}
}
\label{tab:testprioresult}
\end{table}

The dependent tests in our subject programs interfere with
all the XXX test prioritization techniques in Figure~\ref{}. This
is because all these techniques implicitly assume that there
are no test dependences in the input test suite. Violation of
this assumption, as happened in real-world unit test suites,
causes undesired output. Further, test prioritization algorithms
usually donot take the potential test dependence into
consideration when reordering the test suite.

\todo{which test prioritization techniques are the most sensitive ones}

\todo{break down the data a bit and how many tests are changing from
passing to failing? or vice versa}

\todo{give specific examples here}


\subsubsection{Impact on Test Selection}

\begin{table}
\centering
\setlength{\tabcolsep}{1.25\tabcolsep}
\begin{tabular}{|l|c|c|}
%\toprule
\hline
\textbf{Subject Program} & S1 (statement-level) & S2 (method-level)  \\
\hline
\multicolumn{3}{|l|}{}  \\
\multicolumn{3}{|l|}{\textbf{Human-written Test Suites}}  \\
\hline
\jt& 0 & 0 \\
XML Security& 0 & 0 \\
Crystal& 0 & 0\\
Synoptic& 0 & 0  \\
JFreechart& 0 & 0  \\
%\bottomrule
\hline
\textbf{Total} & 0 & 0  \\
\hline
\multicolumn{3}{|l|}{}  \\
\multicolumn{3}{|l|}{\textbf{Automatically-generated Test Suites}}  \\
\hline
\jt& 64 & 261 \\
XML Security& 24 & 24  \\
Crystal& 0 & 0  \\
Synoptic& 0 & 0  \\
JFreechart& 0 & 3  \\
%\bottomrule
\hline
\textbf{Total} & 88 & 288 \\
\hline
%\textbf{Total}& &  & &  \\ 
%\hline
\end{tabular}
\caption{Results of evaluating the \selnum test selection techniques
in Figure~\ref{tab:testprio} on four human-written unit test suites.
Each cell shows the number of dependent tests
that do not return the same results as they do when executed
in the original test suite. \todo{revise the number here}
}
\label{tab:testselresult}
\end{table}

\subsubsection{Impact on Test Parallelization}

\begin{table*}
\centering
\setlength{\tabcolsep}{1.25\tabcolsep}
\begin{tabular}{|l| l|l|l|l| l|l|l|l| l|l|l|l|}
%\toprule
\hline
\textbf{Subject Program} & \multicolumn{4}{|l|}{P1 (Original Order Parallelization)} &  \multicolumn{4}{|l|}{P2 (Random Parallelization)} & \multicolumn{4}{|l|}{P3 (Time-Minimized Parallelization)}\\
\cline{2-13}
& k=2 & k=4 & k=8 & k=16 & k=2 &k=4& k=8& k=16 & k=2 &k=4& k=8& k=16\\
\hline
\multicolumn{13}{|l|}{}  \\
\multicolumn{13}{|l|}{\textbf{Human-written Test Suites}}  \\
\hline
\jt& 0 & 0 & 0 & 0 & 1 & 1 & 1 & 1 & 3 & 1 & 0 & 1\\
XML Security& 0 & 0 & 4 & 4 & 0  & 0 & 0 & 2& 2 & 4 & 4& 4\\
Crystal& 8 & 8 & 18  & 15 & 17 &16& 18 & 18& 9 & 16& 15& 18\\
Synoptic& 0 & 0 & 0 & 0 & 0 & 0 & 0 & 0 & 0 & 0 & 0& 0\\
JFreechart& 0 & 0 & 0 & 0 & 1 &1 & 0 & 0& 2 & 0 & 1 & 1\\
%\bottomrule
\hline
\textbf{Total} & 8 & 8 & 22  & 19  & 19 & 18& 19& 21& 16& 21& 20& 24\\
\hline
\multicolumn{13}{|l|}{}  \\
\multicolumn{13}{|l|}{\textbf{Automatically-generated Test Suites}}  \\
\hline
\jt& 22 & 48 & 201 & 236 & 324 & 267 & 300 & 324& 354 & 268 & 305 & 307\\
XML Security& 66 & 78 & 104 & 111 & 91 & 103 & 111 & 116& 73 & 115 & 113 & 116\\
Crystal& 0  & 0 & 1 & 4 & 93 & 86 & 81 & 91& 88 & 88 & 86 & 102\\
Synoptic& 1 & 1 & 2 & 2 & 2 & 1 & 2 & 1& 2 & 2 & 2 & 2 \\
JFreechart& 1 & 3 & 3 & 3 & 2 & 4 & 3 & 3 & 2 & 2 & 2 & 3\\
%\bottomrule
\hline
\textbf{Total} & 90 & 130  & 311  & 356 & 512 & 461 & 497& 535 & 519& 475& 508& 530\\
\hline
%\textbf{Total}& &  & &  \\ 
%\hline
\end{tabular}
\caption{Results of evaluating the \parnum test parallelization techniques
in Figure~\ref{tab:testprio} on four human-written unit test suites.
Each cell shows the number of dependent tests
that do not return the same results as they do when executed
in the original test suite. \todo{revise the number here}
}
\label{tab:testparresult}
\end{table*}

\subsection{Discussion}

\subsubsection{Root Causes of Test Dependence}

Essentially, test dependence results from
interactions with other tests, as reflected in
the execution environment. Tests may make
implicit assumptions about their execution environment --
values of global variables, contents of files, etc. A dependent
test manifests when another alters the execution environment
in a way that invalidates those assumptions.

Our previous study~\cite{} suggested three common
root causes of test dependence: (1) improper access
to shared global variables (i.e., static
variables in Java), (2) improper access
to the file systems, and (3) improper access
to other external resources (e.g., databases,
networks). Improper access to
shared global variables is the most common
root causes, accounting for at least 61\% of 
all dependent tests we have studied. All dependent
tests in our experiments arise due to access
to static Java variables.

The downstream testing techniques we have evaluated
have indirectly changed the execution environment
that a test may implicitly assume, such as alternating
the test execution order (as test prioritization
does) and selecting a subset of
test to execute (as test selection and parallelization
do).



\subsubsection{Threats to Validity}

Our findings apply in the context of our
study and methodology and may not apply to arbitrary
programs. First, the applications we studied are all written
in Java and have JUnit test suites.
The XXX open-source programs and
their test suites may not be representative enough.
However, these are the first XXX subject programs we
tried, and the fact that we found dependent tests in all of
them and these dependent tests affect all evaluated
downstream testing techniques is suggestive.
Second, in our study, we only evaluated XXX
test prioritization, XXX test selection, and
XXX test parallelization techniques. Evaluating
dependent tests on other testing techniques
may yield different results.

\subsubsection{Experimental Conclusion}

We have three chief findings. \textbf{(1)}
Dependent tests in both human-written
and automatically-generated test suites
can affect the results of test prioritization,
test selection, and test parallelization
techniques. \textbf{(2)} The degree of 
impact varies across different testing
techniques. \todo{explain more}
\textbf{(3)} \todo{talk about the reason}
