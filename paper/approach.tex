\section{Approach}


\begin{comment}
The goal of the paper is to determine whether test dependence can affect test prioritization, to study the extent of the impact, and how we should augment these existing techniques to deal with dependent tests. 

Coverage elements are either statements or functions. Given a test suite the algorithm below gets a possible execution order by counting the number of coverage elements each test executed (line 3-11), sort the tests based on the number of coverage elements it executed (line 12) and returns an order for the tests where the test that executed the most coverage elements is first while the test that executed the least amount is last (line 13-16). In the event that multiple tests execute the same number of coverage elements, those tests will be added in an arbitrary order to the returned list. 

\textbf{two algorithms here are omitted, need to re-write}

We then run the tests with the newly generated order and compare the results of that with the results of running the tests in its original intended order~\cite{}. An original intended order is defined to be any order of a test suite in which all tests pass. If no such order can be found, an arbitrary order is generated by parsing all the files of the project for its test cases and that order is used instead. First we execute T with its original intended order and save the results of that execution as the expected result of each test (line 2). Then we generate a test prioritization execution order (line 3), and execute each test again to observe its result (line 4). The algorithm concludes by checking whether the result of any test differs from the expected result (lines 5--8).

\end{comment}
