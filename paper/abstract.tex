\begin{abstract}

In a test suite, all the test cases should be independent:
no test should affect any other test's result, and running
the tests in any order should produce the same test results.
The assumption of test independence is important so that
tests behave consistently as designed. In addition, many
downstream testing techniques, including test prioritization,
test selection, and test parallelization,
assume test independence. However, as shown in our recent work,
this critical assumption often does not hold in practice. 
%the impact
%of test dependence is still unclear.

%This paper addresses the following question: how is the impact
%of test dependence on downstream testing techniques?
%We 
This paper empirically investigates the impact of test dependence
on three downstream testing techniques (i.e., test
selection, prioritization, and parallelization) and propose algorithms
to cope with such impact. It presents three results.

First, we describe an empirical study to assess the impact of
test dependence
on \prionum test prioritization, \selnum test selection, and
\parnum test parallelization
algorithms. The study suggests that test dependence
affects the results of \textit{all} evaluated downstream
testing algorithms in an nonignorable manner.

Second, we present a family of algorithms to 
enhance each test prioritization, selection, and parallelization
technique to respect test dependence, so that each test
yields the same result before and after applying
the downstream testing technique.
%, ensuring they output
%consistent results.
%The first technique monitors the
%test execution of a test suite and generates a concise report
%to describe the root cause of test dependence.
%The report permits developers understand why and how
%test dependence arises.
%The second technique enhances existing test prioritization,
%test selection, and test parallelization algorithms to ensure
%they respect the test dependence.

Third, we evaluate all enhanced testing techniques and
show that they %describe an experimental evaluation 
%to show that all enhanced testing techniques
output consistent results in the presence of
test dependence without compromising their effectiveness.

%techniques for coping with
%the test dependence impact are useful and effective. They
%help developers understand why and how test dependence arises
%and enable downstream techniques output consistent results
%in the presence of test dependence.

\end{abstract}

